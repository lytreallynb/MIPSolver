
\documentclass[12pt,a4paper]{article}
\usepackage{xeCJK}
\usepackage{amsmath}
\usepackage{amsfonts}
\usepackage{amssymb}
\usepackage{geometry}
\usepackage{booktabs}
\usepackage{array}
\usepackage{longtable}
\usepackage{graphicx}
\usepackage{xcolor}
\usepackage{fancyhdr}

% 设置页面边距
\geometry{margin=2.5cm}

% 设置中文字体 (使用系统字体)
\setCJKmainfont{PingFang SC}  % macOS系统字体
\setCJKsansfont{PingFang SC}
\setCJKmonofont{PingFang SC}

% 如果系统没有PingFang SC,可以使用其他中文字体
% \setCJKmainfont{SimSun}  % Windows
% \setCJKmainfont{Noto Sans CJK SC}  % Linux

% 设置页眉页脚
\pagestyle{fancy}
\fancyhf{}
\fancyhead[L]{MIPSolver 求解报告}
\fancyhead[R]{2025年07月29日 15:33:21}
\fancyfoot[C]{\thepage}

% 标题设置
\title{\textbf{MIPSolver 混合整数规划求解报告}}
\author{系统自动生成}
\date{2025年07月29日 15:33:21}

\begin{document}

\maketitle
\thispagestyle{fancy}

\tableofcontents
\newpage

\section{问题概述}

\subsection{数学模型}

本问题为混合整数线性规划问题,数学模型如下:

\begin{align}
\text{目标函数:} \quad & \min \sum_{j=1}^{n} c_j x_j \label{eq:objective}\\
\text{约束条件:} \quad & \sum_{j=1}^{n} a_{ij} x_j \leq b_i, \quad i = 1, 2, \ldots, m \label{eq:constraints}\\
& x_j \geq 0, \quad j = 1, 2, \ldots, n \label{eq:nonnegativity}\\
& x_j \in \mathbb{Z}, \quad j \in I \label{eq:integrality}
\end{align}

其中:
\begin{itemize}
\item $x_j$ 为决策变量,$j = 1, 2, \ldots, n$
\item $c_j$ 为目标函数系数
\item $a_{ij}$ 为约束系数矩阵元素
\item $b_i$ 为约束右端常数
\item $I$ 为整数变量的指标集合
\end{itemize}

\section{求解结果}

\subsection{求解状态信息}

\begin{table}[h]
\centering
\begin{tabular}{ll}
\toprule
\textbf{项目} & \textbf{结果} \\
\midrule
求解状态 & OPTIMAL \\
目标函数值 & 387.853012 \\
求解时间 & 0.0003 秒 \\
迭代次数 & 20 \\
使用求解器 & Branch \& Bound \\
变量数量 & 24 \\
\bottomrule
\end{tabular}
\caption{求解状态信息汇总}
\end{table}

\subsection{最优解}

求解得到的最优解如下:

\begin{longtable}{lcc}
\toprule
\textbf{变量名} & \textbf{最优值} & \textbf{变量类型} \\
\midrule
X0 & 0.194654 & 连续 \\
X1 & 7.279864 & 连续 \\
X2 & 2.525481 & 连续 \\
X3 & 6.587987 & 连续 \\
X4 & 14.236565 & 连续 \\
X5 & 9.175448 & 连续 \\
X6 & 9.838832 & 连续 \\
X7 & 17.682443 & 连续 \\
X8 & 12.478725 & 连续 \\
X9 & 3.378527 & 连续 \\
X10 & 10.801128 & 连续 \\
X11 & 5.820345 & 连续 \\
Y0 & 0.019465 & 连续 \\
Y1 & 0.727986 & 连续 \\
Y2 & 0.252548 & 连续 \\
Y3 & 0.329399 & 连续 \\
Y4 & 0.474552 & 连续 \\
Y5 & 0.305848 & 连续 \\
Y6 & 0.491942 & 连续 \\
Y7 & 0.442061 & 连续 \\
\midrule
\multicolumn{3}{c}{\textit{... 省略其余 4 个变量 ...}} \\

\bottomrule
\caption{决策变量最优取值}
\end{longtable}

\section{问题分析}

\subsection{问题规模分析}

\begin{table}[h]
\centering
\begin{tabular}{lr}
\toprule
\textbf{问题特征} & \textbf{数量} \\
\midrule
决策变量总数 & 24 \\
连续变量 & 24 \\
整数变量 & 0 \\
二进制变量 & 0 \\
约束数量 & 19 \\
\bottomrule
\end{tabular}
\caption{问题规模统计}
\end{table}

\subsection{求解器性能分析}

\begin{itemize}
\item \textbf{使用求解器:}Branch \& Bound
\item \textbf{求解算法:}基于C++实现的分支定界法
\item \textbf{线性松弛:}单纯形法
\item \textbf{求解效率:}0.0003秒完成求解
\item \textbf{迭代收敛:}经过20次迭代达到最优解
\end{itemize}

\subsection{解的质量评估}

根据求解状态 \textbf{OPTIMAL},可以得出以下结论:

\begin{itemize}
\item 问题具有可行解,求解器成功找到最优解
\item 目标函数最优值为 387.853012
\item 所有约束条件均得到满足
\item 整数变量取值符合整数约束要求
\end{itemize}

\section{总结与结论}

\subsection{求解总结}

本次优化求解任务已成功完成,主要成果如下:

\begin{enumerate}
\item \textbf{问题建模:}成功构建了包含 24 个决策变量和 19 个约束的混合整数线性规划模型
\item \textbf{算法求解:}采用高性能C++实现的分支定界算法,确保求解的准确性与效率
\item \textbf{最优解获得:}在 0.0003 秒内找到最优解,目标函数值为 387.853012
\item \textbf{解的验证:}所有约束条件均得到满足,整数约束得到严格执行
\end{enumerate}

\subsection{技术说明}

\begin{itemize}
\item \textbf{软件平台:}MIPSolver v1.0 - 基于Python和C++的混合整数规划求解器
\item \textbf{求解引擎:}自主研发的高性能C++优化核心
\item \textbf{报告生成:}支持XeLaTeX格式,完美呈现中文内容
\item \textbf{生成时间:}2025年07月29日 15:33:21
\end{itemize}

\vspace{1cm}

\begin{center}
\textit{--- 报告结束 ---}

\small{此报告由 MIPSolver 系统自动生成}
\end{center}

\end{document}
